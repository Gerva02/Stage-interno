\chapter{Introduzione}
\label{sect:History Causality}
Il problema della causalità è di fondamentale interesse per capire meglio il mondo che ci circonda, perciò la domanda è stata affrontata da  filosofi e scienziati. La sua definizione rigorosa nel campo statistico è stata solo affrontata molto recentemente, [proprio perché estremamente vago], da Neyman (1932) e poi sviluppata da Rubin negli anni '70, i quali hanno fatto riferimento al modello dei potential outcome spiegato nel capitolo \ref{sect:PotentialOM}. Questo modello diverge concettualmente da alcune definizioni portate avanti da  filosofi come John Stuart Mill che definisce la causalità come “the antecedent, or the concurrence of antecedents, on which [a given phenomenon] is invariably and unconditionally consequent”, per Mill dunque possiamo dire che A causa B se e solo se ogni volta che succede A succede anche B . Questo modello di causalità è molto riduttivo e ignora la aleatorietà degli eventi, per questo dobbiamo introdurre il potential outcome model. 