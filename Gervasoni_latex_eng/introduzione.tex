\chapter{Introduzione}
\label{chapt:History Causality}
The problem of causality is of fundamental interest and it has been for as long as humans have been around. We always have asked why something happens and we tried many possible answers. The question of causality has been explored by philosopher and scientist alike, but its precise statistical definition has remained elusive until the seventies, when Rubin expanded and perfected a theory proposed by Neyman in 1932. This model diverges in a few key points from the definition of some philosopher; John Stuart Mill defines causality as  “the antecedent, or the concurrence of antecedents, on which [a given phenomenon] is invariably and unconditionally consequent”. Mill thinks that A causes B if and only if each time A happens than B happens. This kind of intuition not only is wrong but can be outright dangerous. Instead in this thesis we rely on the potential outcome model that we believe gives more satisfactory answers to the aleatory nature of reality.
